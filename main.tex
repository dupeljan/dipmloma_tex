\documentclass[oneside,final,14pt]{extreport}

%% my command
%%%%%%%%%%%%%%
% Путь к файлу с изображениями
\newcommand{\picPath}{pictures}
% Величина отступа
\newcommand{\indentSpace}{1.25cm}
% Сокращения
\newcommand{\urlTitle}{ $-$ URL: }
%%%%%%%%%%%%%%%


% Изменяем шрифт
\usepackage{fontspec}
\setmainfont{Times New Roman}
\listfiles

% Полуторный интервал
\linespread{1.6}

% Отступ
\setlength\parindent{\indentSpace}

% Математика
\usepackage{mathtools}


% Картинки
\usepackage{graphicx}
\usepackage{subcaption}

% Языковой пакет
\usepackage[russianb]{babel}

% Таблицы
\usepackage{tabularx}

% Настройка подписей к фигурам
% Меняем заголовки картинок
\usepackage[ labelsep= endash]{caption}
\captionsetup{%
   figurename= Рисунок,
   tablename= Таблица,
   justification= centering% Формат - по центру
}         

% Кирилица в подфигурах
\renewcommand{\thesubfigure}{\asbuk{subfigure}}
% разделитель в подфигурах - правая скобка
\DeclareCaptionLabelSeparator{r_paranthesis}{)\quad }
\captionsetup[subfigure]{labelformat=simple, labelsep=r_paranthesis}

% Добавляем итератор \asbuk,
% чтобы использовать кирилицу
% как маркеры
\usepackage{enumitem}
\makeatletter
\AddEnumerateCounter{\asbuk}{\russian@alph}{щ}
\makeatother

% Меняем маркеры в перечислениях
% Списки уровня 1
\setlist[enumerate,1]{label=\arabic*),ref=\arabic*}
% Списки уровня 2
\setlist[enumerate,2]{label=\asbuk*),ref=\asbuk*}
% Перечисления
\setlist[itemize,1]{label=$-$}
% Удаляем отступы перед и после
% списка
\setlist[itemize]{noitemsep, topsep=0pt}
\setlist[enumerate]{noitemsep, topsep=0pt}

% Красная строка в начале главы
\usepackage{indentfirst}

% Убиваем перенос
\usepackage[none]{hyphenat}

% Перенос длинных ссылок
\usepackage[hyphens]{url}
\urlstyle{same}

% Выравнивание по ширине
\usepackage{microtype}

%\usepackage[fontfamily=courier]{fancyvrb}
%\usepackage{verbatim}%     configurable verbatim
% \makeatletter
%  \def\verbatim@font{\normalfont\sffamily% select the font
%                     \let\do\do@noligs
%                     \verbatim@nolig@list}
%\makeatother

% Границы
\usepackage{vmargin}
\setpapersize{A4}
% отступы
%\setmarginsrb 
%{3cm} % левый
%{2cm} % верхний
%{1cm} % Правый
%{2cm} % Нижний
%{0pt}{0mm} % Высота - отступ верхнего колонтитула
%{0pt}{0mm} % Высота - отступ нижнего  колонтитула

\setlength\hoffset{0cm}
\setlength\voffset{0cm}
\usepackage[top=2cm, bottom=2cm, left=3cm, right=2cm,
]{geometry}
 		
% Настройка заглавиий
\addto\captionsrussian{% Replace "english" with the language you use
  \renewcommand{\contentsname}% содержания
    {\hfill\bfseries
    СОДЕРЖАНИЕ
	\hfill    
    }%
   \renewcommand{\bibname}% списка источников
    {\hfill\bfseries
    	СПИСОК ИСПОЛЬЗОВАННЫХ ИСТОЧНИКОВ
	\hfill
	}% 
}%\

%\renewcommand{\contentsname}{\hfill\bfseries СОДЕРЖАНИЕ \hfill} 

% Настройка  заглавий в главах
\usepackage{titlesec}


%\titleformat
%{\chapter} % command
%[display]
%{
%\bfseries
%} % format
%{
%\thechapter.
%} 	% label
%{ 
%	0 pt
%} % sep
%{    
%\centering
%} % before-code

\titleformat{\chapter}
            {\bfseries}
            {\hspace{\indentSpace}\thechapter\hspace{1em}}
            {0pt}
            {
            \vspace{0mm} }
            [\vspace{14pt}]% Отступ после
% Начальный сдвиг заголовка 50 pt = 1.763888888cm.
% Второй параметр- сдвиг до = 2cm - 50pt
\titlespacing{\chapter}{0pt}{-0.2361cm}{0pt}

\titleformat{\section}
{\bfseries}{\hspace{\indentSpace}\thesection}{1em}{}

\titleformat{\subsection}
{\bfseries}{\hspace{\indentSpace}\thesubsection}{1em}{}

%\titleformat{\section}
%            {\bfseries}
%            {\thechapter.\hspace{1em}}
%            {0pt}
%            {\centering
%            \vspace{0mm} }
%            [\vspace{14pt}]% Отступ после
%\titlespacing{\section}{0pt}{-50pt}{0pt}

% Конец настройка заглавий

% Форматирование списка источников
\makeatletter
\renewcommand*{\@biblabel}[1]{\hfill#1}
\makeatother

% Убрать отсупы в списке источников
\usepackage{lipsum}

% ADD THE FOLLOWING COUPLE LINES INTO YOUR PREAMBLE
\let\OLDthebibliography\thebibliography
\renewcommand\thebibliography[1]{
  \OLDthebibliography{#1}
  \setlength{\parskip}{0pt}
  \setlength{\itemsep}{0pt plus 0.3ex}
}



% Добавить точки в оглавление
\usepackage{tocstyle}
\newcommand{\autodot}{.}


% Чтобы картинки вставляись
% куда надо
\usepackage{float}

% Для вычисления кол-ва страниц
\usepackage{lastpage}

% Для вычисления кол-ва рисунков и таблиц
%%%
\usepackage{etoolbox}

\newcounter{totfigures}
\newcounter{tottables}

\providecommand\totfig{} 
\providecommand\tottab{}

\makeatletter
\AtEndDocument{%
  \addtocounter{totfigures}{\value{figure}}%
  \addtocounter{tottables}{\value{table}}%
  \immediate\write\@mainaux{%
    \string\gdef\string\totfig{\number\value{totfigures}}%
    \string\gdef\string\tottab{\number\value{tottables}}%
  }%
}
\makeatother

\pretocmd{\chapter}{\addtocounter{totfigures}{\value{figure}}\setcounter{figure}{0}}{}{}
\pretocmd{\chapter}{\addtocounter{tottables}{\value{table}}\setcounter{table}{0}}{}{}
%%%

% Режим релиза
\sloppy
\usepackage{layout}

%\renewcommand{\arraystretch}{1.6}

\newcommand{\cmmnt}[1]{\ignorespaces}
\newcommand{\bs}{\boldsymbol}
\usepackage{breqn}
\begin{document}
\begin{center}
\bfseries РЕФЕРАТ
\end{center}

Дипломная работа содержит \pageref{LastPage} страниц, \totfig\ рисунков, \tottab\ таблицы, 5 источников.

ВЕБИНАР, КОРРЕЛЯЦИЯ ИЗОБРАЖЕНИЙ, ПРЕОБРАЗОВАНИЕ ФУРЬЕ, ЛОКАЛИЗАЦИЯ ТЕКСТА, TESSERACT, РАСПОЗНАВАНИЕ ТЕКСТА

Объектом исследования являются мобильные роботы, использующие роликонесущие колеса и методы управления ими.

Цель курсовой работы $-$ программная реализация процедур, вычисляющих основные критерии оценки вебинаров.

В результате работы были реализованы процедуры, реализующие
\begin{itemize}
\item поиск указки;
\item выделение слайдов;
\item распознавание блоков текста на слайдах.
\end{itemize} 

\tableofcontents
\newpage
\begin{center}
\bfseries ВВЕДЕНИЕ
\end{center}
\addcontentsline{toc}{chapter}{Введение}

Перед автором была поставлена задача $-$ исследовать виды движущихся роботизированных систем и методы управления ими. Особенный интерес изучения представляют роботизированные системы, использующие для передвижения роликонесущие колеса. Этот тип колес широко используется при создании роботизированных систем и позволяет роботам двигаться в любом заданном направлении в плоскости движения. Это увеличивает область применения роботизированных систем: в малых помещениях, в которых роботизированным системам на обычных колесах не хватает места для передвижения, роботы на роликонесущих колесах способны передвигаться без каких-либо ограничений. 

В процессе выполнения дипломной работы были разработаны модель роликонесущего колеса, механическая и кинематичесая модели тележек, опирающихся на $N$ роликонесущих колес, $N>2$, изучены и обозрены методы управления роботизированными системами. Полученные сведения протестированы на виртуальной модели робота с учетом всех физических сил, воздействующих на систему.

\chapter{Классификация мобильных роботов}
Мобильным роботом называют робота, способного менять свое местоположение в пространстве. Мобильные роботы могут быть автономными и управляемыми вручную. Автономные мобильные роботы способны без участия человека, основываясь на показаниях установленных на нем сенсоров и датчиков, определять свое местоположение и окружение, в котором они находится. Управляемый вручную робот не имеет такую возможность и способен передвигаться только по заранее заданной траектории.

Для того, чтобы передвигаться в пространстве, мобильный робот должен иметь в своем устройстве механизм, приводящий его в движение. Мобильные роботы способны передвигаться используя следующие техники:
\begin{itemize}
\item ходьба;
\item прыжки;
\item скольжение;
\item качение;
\item плавание;
\item полет;
\item кувырки.
\end{itemize}
[ссылка на книжку]
Естественно, техники могут комбинироваться. 
 В рамках работы исследуются механизмы, позволяющие роботу двигаться по твердым горизонтальным поверхностям в земной среде. Кроме того, ограничим возможные техники движения ходьбой и качением Существующие роботы, способные двигаться по горизонтальной плоскости, делятся на следующие категории:
\begin{itemize}
\item роботы, использующие ноги для движеня;
\item роботы, использующие колеса для движения;
\item роботы, использующие гусеницы для движения.
\end{itemize} 
\section{Роботы, использующие ноги для движения}
Способ движения существующих роботов, использующих ноги, во многом повторяет способы предевижения биологических существ. Роботы этого типа имеют больше степеней своботы в сравнении с колесными роботами, что делает их устройство гораздо сложнее.

Роботы, использующие ноги, используются в условиях, когда поверхность движения не является плоской или материал поверхности мягкий. Во время качения по плоской твердой поверхности колесо имеет малую площадь соприкасновения с поверхностью, поэтому при качении колесо испытывает малое количество сопротивления. Неровности и мягкий материал поверхности увеличивает площадь поверхности колеса и уменьшает его эффективность. Для создания условий движения колеса требуется большое количество ограничений. Роботы, использующие ноги для движения, в отличие от колесных, имеют большую площадь соприкасновения с поверхностью, что дает им преимущество в сложных условиях.

Роботы, использующие ноги для движения способны прередвигаться в сложных условиях, когда поверхность не является ровной, поверхность подъемы или спуски или состоит из мягкого материала. Это позволяет широко использовать таких роботов в сложной среде обитания. К недостаткам роботов с ногами можно отнести высокую сложность механизма и сравнительно низкую скорость передвижения.
\section{Роботы, использующие колеса для движения}
блаблабла
\subsection{Колесо 1}
\subsection{Колесо 2}
\subsection{Колесо 3}
\section{Роботы, использующие гусеницы для движения}
\end{document}