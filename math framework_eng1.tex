\documentclass[oneside,final,14pt]{extreport}

%% my command
%%%%%%%%%%%%%%
% Путь к файлу с изображениями
\newcommand{\picPath}{pictures}
% Величина отступа
\newcommand{\indentSpace}{1.25cm}
% Сокращения
\newcommand{\urlTitle}{ $-$ URL: }
%%%%%%%%%%%%%%%


% Изменяем шрифт
\usepackage{fontspec}
\setmainfont{Times New Roman}
\listfiles

% Полуторный интервал
\linespread{1.6}

% Отступ
\setlength\parindent{\indentSpace}

% Математика
\usepackage{mathtools}


% Картинки
\usepackage{graphicx}
\usepackage{subcaption}

% Языковой пакет
\usepackage[russianb]{babel}

% Таблицы
\usepackage{tabularx}

% Настройка подписей к фигурам
% Меняем заголовки картинок
\usepackage[ labelsep= endash]{caption}
\captionsetup{%
   figurename= Рисунок,
   tablename= Таблица,
   justification= centering% Формат - по центру
}         

% Кирилица в подфигурах
\renewcommand{\thesubfigure}{\asbuk{subfigure}}
% разделитель в подфигурах - правая скобка
\DeclareCaptionLabelSeparator{r_paranthesis}{)\quad }
\captionsetup[subfigure]{labelformat=simple, labelsep=r_paranthesis}

% Добавляем итератор \asbuk,
% чтобы использовать кирилицу
% как маркеры
\usepackage{enumitem}
\makeatletter
\AddEnumerateCounter{\asbuk}{\russian@alph}{щ}
\makeatother

% Меняем маркеры в перечислениях
% Списки уровня 1
\setlist[enumerate,1]{label=\arabic*),ref=\arabic*}
% Списки уровня 2
\setlist[enumerate,2]{label=\asbuk*),ref=\asbuk*}
% Перечисления
\setlist[itemize,1]{label=$-$}
% Удаляем отступы перед и после
% списка
\setlist[itemize]{noitemsep, topsep=0pt}
\setlist[enumerate]{noitemsep, topsep=0pt}

% Красная строка в начале главы
\usepackage{indentfirst}

% Убиваем перенос
\usepackage[none]{hyphenat}

% Перенос длинных ссылок
\usepackage[hyphens]{url}
\urlstyle{same}

% Выравнивание по ширине
\usepackage{microtype}

%\usepackage[fontfamily=courier]{fancyvrb}
%\usepackage{verbatim}%     configurable verbatim
% \makeatletter
%  \def\verbatim@font{\normalfont\sffamily% select the font
%                     \let\do\do@noligs
%                     \verbatim@nolig@list}
%\makeatother

% Границы
\usepackage{vmargin}
\setpapersize{A4}
% отступы
%\setmarginsrb 
%{3cm} % левый
%{2cm} % верхний
%{1cm} % Правый
%{2cm} % Нижний
%{0pt}{0mm} % Высота - отступ верхнего колонтитула
%{0pt}{0mm} % Высота - отступ нижнего  колонтитула

\setlength\hoffset{0cm}
\setlength\voffset{0cm}
\usepackage[top=2cm, bottom=2cm, left=3cm, right=2cm,
]{geometry}
 		
% Настройка заглавиий
\addto\captionsrussian{% Replace "english" with the language you use
  \renewcommand{\contentsname}% содержания
    {\hfill\bfseries
    СОДЕРЖАНИЕ
	\hfill    
    }%
   \renewcommand{\bibname}% списка источников
    {\hfill\bfseries
    	СПИСОК ИСПОЛЬЗОВАННЫХ ИСТОЧНИКОВ
	\hfill
	}% 
}%\

%\renewcommand{\contentsname}{\hfill\bfseries СОДЕРЖАНИЕ \hfill} 

% Настройка  заглавий в главах
\usepackage{titlesec}


%\titleformat
%{\chapter} % command
%[display]
%{
%\bfseries
%} % format
%{
%\thechapter.
%} 	% label
%{ 
%	0 pt
%} % sep
%{    
%\centering
%} % before-code

\titleformat{\chapter}
            {\bfseries}
            {\hspace{\indentSpace}\thechapter\hspace{1em}}
            {0pt}
            {
            \vspace{0mm} }
            [\vspace{14pt}]% Отступ после
% Начальный сдвиг заголовка 50 pt = 1.763888888cm.
% Второй параметр- сдвиг до = 2cm - 50pt
\titlespacing{\chapter}{0pt}{-0.2361cm}{0pt}

\titleformat{\section}
{\bfseries}{\hspace{\indentSpace}\thesection}{1em}{}

\titleformat{\subsection}
{\bfseries}{\hspace{\indentSpace}\thesubsection}{1em}{}

%\titleformat{\section}
%            {\bfseries}
%            {\thechapter.\hspace{1em}}
%            {0pt}
%            {\centering
%            \vspace{0mm} }
%            [\vspace{14pt}]% Отступ после
%\titlespacing{\section}{0pt}{-50pt}{0pt}

% Конец настройка заглавий

% Форматирование списка источников
\makeatletter
\renewcommand*{\@biblabel}[1]{\hfill#1}
\makeatother

% Убрать отсупы в списке источников
\usepackage{lipsum}

% ADD THE FOLLOWING COUPLE LINES INTO YOUR PREAMBLE
\let\OLDthebibliography\thebibliography
\renewcommand\thebibliography[1]{
  \OLDthebibliography{#1}
  \setlength{\parskip}{0pt}
  \setlength{\itemsep}{0pt plus 0.3ex}
}



% Добавить точки в оглавление
\usepackage{tocstyle}
\newcommand{\autodot}{.}


% Чтобы картинки вставляись
% куда надо
\usepackage{float}

% Для вычисления кол-ва страниц
\usepackage{lastpage}

% Для вычисления кол-ва рисунков и таблиц
%%%
\usepackage{etoolbox}

\newcounter{totfigures}
\newcounter{tottables}

\providecommand\totfig{} 
\providecommand\tottab{}

\makeatletter
\AtEndDocument{%
  \addtocounter{totfigures}{\value{figure}}%
  \addtocounter{tottables}{\value{table}}%
  \immediate\write\@mainaux{%
    \string\gdef\string\totfig{\number\value{totfigures}}%
    \string\gdef\string\tottab{\number\value{tottables}}%
  }%
}
\makeatother

\pretocmd{\chapter}{\addtocounter{totfigures}{\value{figure}}\setcounter{figure}{0}}{}{}
\pretocmd{\chapter}{\addtocounter{tottables}{\value{table}}\setcounter{table}{0}}{}{}
%%%

% Режим релиза
\sloppy
\usepackage{layout}


%\renewcommand{\arraystretch}{1.6}

\begin{document}
\chapter{Kinematic model of a robot with N omnidirectional wheels}
In order to describe the movements of the robot, it is necessary to model its behavior in some way. The simplest model of the movement of a robot in space is a kinematic one. This model describes movements exclusively through the dependence of coordinates on time. That is, in the kinematic model, the movement of the body is considered, but the reasons that create it are not considered.
Consider the movement of a cart with $ N $ omnidirectional wheels ($ N> 3 $) on a smooth two-dimensional surface without taking into account the acting forces, and the plane of the cart wheels is vertical and stationary relative to the cart platform. Within the model, omnidirectional wheels can slide in any direction with a negligible friction force. Let a global coordinate system associated with the surface $ [X, Y] $ and a local, inertial relative to the global one, rigidly connected with the cart $ [X_{l}, Y_{l}] $ be given. Without loss of generality, the origin of local coordinates at the point of the center of mass of the cart. The position of the cart is determined by the coordinate vector $ (x, y, \varphi) $
where $ x, y $ are the coordinates, and $ \varphi $ is the angle of rotation of the coordinate system $ [X_{l}, Y_{l}] $ relative to the axis $ OY $ in the global coordinate system. The speed of the cart is determined by the vector $ (\dot{x}, \dot{y}, \omega) $ where $ \omega = \dot{\varphi} $ is the angular velocity of the cart.

For example, the figure \ref{fig: kinematic_model} schematically shows a kinematic model of a robot with three omnidirectional wheels.
\begin{figure} [H]
\centering{
\resizebox{140mm}{!}{\input{kinematic_model_pic.pdf_tex}}
\caption{kinematic model of a cart with three omnidirectional wheels.}
\label{fig: kinematic_model}
}
\end{figure}

Consider the speed value of the $ i $ th wheel, $ v_{i} $, $ i = \overline{1, N} $. We decompose it into translational and rotational components:
\begin{equation}
v_{i}
=
v_{i, tr} + v_{i, rot}
\end{equation}

We denote the $ \alpha_{i} $-angle between the tangent to the disk of the $ i $ -th wheel and the axis $ OY_{l} $, which is shown in the figure
\ref{fig: projection}.
\begin{figure} [H]
\centering{
\resizebox{60mm}{!}{\input{projection.pdf_tex}}
\caption{location of model vectors in the plane}
\label{fig: projection}
}
\end{figure}

The translational velocity vector makes the angle $ \varphi + \alpha_{i} $ with the $ Y $ axis. Then the value of the translational motion speed can be represented as

\begin{equation}
v_{i, tr}
=
-sin (\varphi + \alpha_{i}) \dot{x}
+ cos (\varphi + \alpha_{i}) \dot{y}
\end{equation}



The value of the speed of rotational motion is given by the following equation
\begin{equation}
v_{i, rot}
=
R_{i} \omega
\end{equation}

Where $ R_{i} $ is the distance from the axis of rotation (center of mass point) to the axis of the $ i $ th wheel.
\iffalse
 Also fair
\begin{gather}
\dot{x}
=
v_{x} \boldsymbol{x}
\\
\dot{y}
=
v_{y} \boldsymbol{y}
\\
\dot{\varphi}
=
\omega \boldsymbol{z}
\end{gather}
Where $ v_{x} $, $ v_{x} $, $ \omega $ is the velocity value in the direction of $ \boldsymbol{x} $ and $ \boldsymbol{y} $, respectively, $ \omega $ is the angular velocity value.
\fi
Thus, the speed value can be written in the following form:
\begin{equation}
v_{i}
=
-sin (\varphi + \alpha_{i}) \dot{x}
+ cos (\varphi + \alpha_{i}) \dot{y}
+
R_{i} \omega
\end{equation}

We correlate the velocity value of the $ i $ - th wheel with its angular velocity
\begin{equation}
v_{i}
=
r_{i} \omega_{i}
\end{equation}

Where $ r_{i} $ is the radius of the $ i $ th wheel of the trolley, $ \omega_{i} $ is its angular velocity, we obtain the dependence of the angular velocity of the wheel on the triple $ (\dot{x}, \dot{y} , \omega) $

\begin{equation}
\omega_{i}
=
\frac{1}{r_{i}}
(
-sin (\varphi + \alpha_{i}) \dot{x}
+ cos (\varphi + \alpha_{i}) \dot{y}
+
R_{i} \omega
)
\end{equation}

or in matrix form

\begin{equation}
\label{kinematic_omniwheel_res}
\begin{bmatrix}
\omega_{1} \\
\omega_{2} \\
... \\
\omega_{N}
\end{bmatrix}
=
\begin{bmatrix}
- \frac{1}{r_{1}} sin (\varphi + \alpha_{1}) &
\frac{1}{r_{1}} cos (\varphi + \alpha_{1}) &
\frac{1}{r_{1}} R_{1}
\\
- \frac{1}{r_{2}} sin (\varphi + \alpha_{2}) &
\frac{1}{r_{2}} cos (\varphi + \alpha_{2}) &
\frac{1}{r_{2}} R_{2}
\\
... & ... & ...
\\
- \frac{1}{r_{N}} sin (\varphi + \alpha_{N}) &
\frac{1}{r_{N}} cos (\varphi + \alpha_{N}) &
\frac{1}{r_{N}} R_{N}
\end{bmatrix}
\begin{bmatrix}
\dot{x} \\
\dot{y} \\
\omega
\end{bmatrix}
\end{equation}
Thus, knowing the coordinates of the path trajectory $ (x, y, \varphi) $, we can obtain the value of the angular velocity of each wheel. To do this, calculate $ (\dot{x}, \dot{y}, \omega) $ and substitute in the formula \ref{kinematic_omniwheel_res}.

In addition, moving to local coordinates

\begin{equation}
\begin{bmatrix}
\dot{x} \\
\dot{y} \\
\omega
\end{bmatrix}
=
\begin{bmatrix}
cos (\varphi) & 0 & 0 \\
0 & cos (\varphi) & 0 \\
0 & 0 & 1
\end{bmatrix}
\begin{bmatrix}
\dot{x_{l}} \\
\dot{y_{l}} \\
\omega
\end{bmatrix}
\end{equation}

We get the equation of movement of the cart in the local coordinate system
\begin{equation}
\begin{bmatrix}
\dot{\phi_{1}} \\
\dot{\phi_{2}} \\
... \\
\dot{\phi_{N}}
\end{bmatrix}
=
\begin{bmatrix}
- \frac{1}{r_{1}} sin (\varphi + \alpha_{1}) &
\frac{1}{r_{1}} cos (\varphi + \alpha_{1}) &
\frac{1}{r_{1}} R_{1}
\\
- \frac{1}{r_{2}} sin (\varphi + \alpha_{2}) &
\frac{1}{r_{2}} cos (\varphi + \alpha_{2}) &
\frac{1}{r_{2}} R_{2}
\\
... & ... & ...
\\
- \frac{1}{r_{N}} sin (\varphi + \alpha_{N}) &
\frac{1}{r_{N}} cos (\varphi + \alpha_{N}) &
\frac{1}{r_{N}} R_{N}
\end{bmatrix}
\begin{bmatrix}
cos (\varphi) & 0 & 0 \\
0 & cos (\varphi) & 0 \\
0 & 0 & 1
\end{bmatrix}
\begin{bmatrix}
\dot{x_{l}} \\
\dot{y_{l}} \\
\omega
\end{bmatrix}
\end{equation}

where $ \dot{x_{l}} $, $ \dot{y_{l}} $ is the velocity in the direction of $ X_{l} $ and $ Y_{l} $, respectively.


\chapter{Dynamic robot model with N omnidirectional wheels}
The dynamic model, in contrast to the kinematic, considers the motion of solids, taking into account the forces that cause this body to move.
 
In order to set the trolley in motion, its wheels need to transmit torque $ T $. By increasing the speed of the trolley, we overcome its resting resistance and give it acceleration.
 
Consider the movement of a trolley with $ N $ omnidirectional wheels ($ N> 3 $) over a nonsmooth two-dimensional surface, taking into account the acting forces. In the description of the model, the same notation is used as in the kinematic model.For example, the figure \ref{fig: dynamic_model} schematically shows a dynamic model with three omnidirectional wheels. Consider the resultant of all the forces acting on the cart $ F $. Force as a measure of the impact on the body characterizes the translational motion of the model, while the moment of forces characterizes the rotational motion. The resultant of all forces can be decomposed into translational and rotational components, similar to the kinematic model, as follows:
\begin{equation}
F
=
\begin{bmatrix}
F_{X} \\
F_{Y} \\
M_{t}
\end{bmatrix}
=
\begin{bmatrix}
M & 0 & 0 \\
0 & M & 0 \\
0 & 0 & J
\end{bmatrix}
\cdot
\begin{bmatrix}
\ddot{x} \\
\ddot{y} \\
\dot{\omega}
\end{bmatrix}
=
B \cdot \ddot{u}
\end{equation}

Where $ M $ is the mass of the trolley, $ J $ is its moment of inertia.

\begin{figure} [H]
\centering{
\resizebox{140mm}{!}{\input{dynamic_model_pic.pdf_tex}}
\caption{dynamic model of a cart with three omnidirectional wheels}
\label{fig: dynamic_model}
}

\end{figure}
The moment of inertia of the trolley can be considered approximately equal to the moment of inertia of a cylinder of radius $ R = max_{i} (R_{i}) $
\begin{equation}
J
=
\frac{1}{2}
MR ^{2}
\end{equation}



Similar to the kinematic model,
projecting the translational force vector onto $ [X, Y] $ using the property of associative moments of forces
we can represent the components of the vector $ F $ as the sum of the projections on the corresponding axes:

\begin{equation}
F_{X}
=
-
\sum_{i = 1} ^{N} f_{i} sin (\varphi + \alpha_{i})
\end{equation}
\begin{equation}
F_{Y}
=
\sum_{i = 1} ^{N} f_{i} cos (\varphi + \alpha_{i})
\end{equation}
\begin{equation}
M_{t}
=
\sum_{i = 1} ^{N} f_{i} R_{i}
\end{equation}
where $ f_{i} $ is the force applied to the top point of the $ i $ th wheel.

Recording in matrix form
\begin{equation}
\begin{bmatrix}
F_{X} \\
F_{Y} \\
M_{t}
\end{bmatrix}
=
\begin{bmatrix}
-sin (\varphi + \alpha_{1}) &
-sin (\varphi + \alpha_{2}) &
... &
-sin (\varphi + \alpha_{N}) \\
cos (\varphi + \alpha_{1}) &
cos (\varphi + \alpha_{2}) &
... &
cos (\varphi + \alpha_{N}) \\
R_{1} & R_{2} & ... & R_{N}
\end{bmatrix}
\begin{bmatrix}
f_{1} \\
f_{2} \\
... \\
f_{n}
\end{bmatrix}
=
Af
\end{equation}

We get the equation
\begin{equation}
F = af
\label{eq: f}
\end{equation}

Solving the equation \ref{eq: f} with respect to $ f $, we obtain
\begin{equation}
f
=
A ^{- 1}
\cdot
F
=
A ^{- 1}
\cdot
B
\cdot
\ddot{u}
\end{equation}

Knowing that the torque of the wheel is equal to the force applied at its top point on its radius and denoting
\begin{equation}
\overline{r}
=
\begin{bmatrix}
r_{1} \\
r_{2} \\
... \\
r_{N}
\end{bmatrix}
\end{equation}
we obtain the dependence of the coordinates of the path $ (x, y, \varphi) $ on the torques of each wheel:
\begin{equation}
\label{eq: dynamic_w_friction}
\begin{bmatrix}
T_{1} \\
T_{2} \\
... \\
T_{N}
\end{bmatrix}
=
\overline{r}
f ^{T}
=
\overline{r}
(
A ^{- 1}
\cdot
B
\cdot
\ddot{u}
) ^{T}
\end{equation}

In turn, the magnitude of the moment of forces of each wheel can be expressed through its angular acceleration:

\begin{equation}
T_{i}
=
J_{i} \dot{\omega_{i}}
\end{equation}

$ J_{i} $ - moment of inertia of the $ i $ th wheel

We take into account the rolling friction force of the model on a certain surface, assuming a uniform distribution of weight on each wheel. It is known that the rolling friction force is opposite in direction to the force that drives the wheel and for the $ i $ -th wheel can be calculated by the formula:
\begin{equation}
f_{t_{i}}
=
\frac{Mg \eta}{Nr_{i}}
\end{equation}

Where $ Mg $ is the support reaction force equal to gravity, $ \eta $ is the rolling friction coefficient, which depends on the surface characteristics.
The rolling friction force is directed in the opposite direction from the direction of movement of the wheel. Therefore, in order to compensate for it, it is necessary to increase the force applied at the upper point of the wheel by the value of the friction force.
\begin{equation}
\tilde{f_{i}}
=
f_{i}
+
f_{i_{t}}
\end{equation}

We denote
\begin{equation}
\tilde{f}
=
f
+
\begin{bmatrix}
f_{t_{1}} \\
f_{t_{2}} \\
... \\
f_{t_{N}}
\end{bmatrix}
=
f
+
\frac{Mg \eta}{N}
\cdot
\begin{bmatrix}
\frac
{1}{r_{1}} \\
\frac{1}{r_{2}} \\
... \\
\frac{1}{r_{N}}
\end{bmatrix}
\end{equation}
Then replacing the formula \ref{eq: dynamic_w_friction} $ f $ with $ \tilde{f} $, we obtain the dependence of the moment of forces of each wheel on the coordinates of the path, taking into account the friction force:
\begin{equation}
\begin{bmatrix}
T_{1} \\
T_{2} \\
... \\
T_{N}
\end{bmatrix}
=
\overline{r}
(
A ^{- 1}
\cdot
B
\cdot
\ddot{u}
+
\frac{Mg \eta}{N}
\cdot
\begin{bmatrix}
\frac{1}{r_{1}} \\
\frac{1}{r_{2}} \\
... \\
\frac{1}{r_{N}}
\end{bmatrix}
) ^{T}
\end{equation}

Thus, we obtain a dynamic model of the trolley, taking into account its mass and rolling friction force.

\iffalse
In ref, the dynamic model of a trolley with N roller-bearing wheels is considered in more detail. The trolley model is built on the assumption that the roller-bearing wheel model is a solid disk, the speed of the lowest point of which is perpendicular to the direction of the roller axis at this point. In addition, the controllability criterion of such a model is considered, from which a number of restrictions on its design follow. For example, if the sum of the angle of the axis of the roller and the angle of rotation of the wheel of all roller-bearing wheels is equal, then such a model cannot move along an arbitrary trajectory. If these angles of the two wheels coincide, then a pair of wheels can be taken as one virtual wheel with the radius vector $ r_{ij} = r_{i} - r_{j} $ in the local coordinate system and the moment $ T_{ij} = T_{i} + T_{j} $.
\fi
\chapter{Kinematic model of a roller-bearing wheel}
For a non-roller wheel that rotates on a surface without slipping, the instantaneous speed of the lowest point of contact is zero [Link to source]. In practice, contacting bodies, due to physical limitations, are always in contact with many points called the contact patch. As an illustration, you can bring a caterpillar trolley moving on the surface without slipping. The caterpillar’s ​​contact spot is large enough to notice that at a certain moment it does not move relative to the surface. At a fairly low speed of the cart, you can even notice that some sub-area of ​​the track’s contact spot does not move relative to the surface for a certain period of time.

The peculiarity of the roller-bearing wheels is that the instantaneous point of contact has a non-zero velocity relative to the rolling surface, since the rollers mounted on the wheel under the gravity of the trolley rotate about its axis. In the case of an omnidirectional wheel, the axis of the rollers of which are parallel to the plane of the wheel disk, slippage does not occur, which simplifies the process of constructing a mathematical model

 The simplest non-holonomic model of a roller-bearing wheel is a flat disk, for which the speed of the point of contact with the bearing surface is directed along a straight line, making a constant angle with the plane of the wheel.
 
 As a model of a roller-bearing wheel, we consider a disk of radius $ R $ centered at the point $ C $, and the planes of the cart wheels are vertical and stationary relative to the cart platform. We denote by $ l $ the unit vector of the axis of rotation of the roller, $ \tau $ is the tangent to the plane of the disk, $ \delta = \widehat{l \tau} $. Then the coupling equation is written as follows:
\begin{equation}
v_{M}
=
v_{C}
+
\omega
\times
\overrightarrow{CM}
\end{equation}
where $ M $ is the point of tangency, $ v_{M} $ is the speed of the point of tangency, $ v_{C} $ is
wheel center speed and $ \omega $ - wheel angular speed.
Let $ \{O, e_{x}, e_{y}, e_{z} \} $ be the global inertial coordinate system, $ \{C, E_{x}, E_{y}, E_{z} \} $ is the coordinate system rigidly connected to the disk. The plane $ \{C, E_{x}, E_{y} \} $ is parallel to the plane of the surface. Let $ \varphi $ be the angle of rotation of the disk around the perpendicular plane of the disk axis passing through the point $ C $ counterclockwise, $ \psi = \widehat{\tau e_{x}} $
\end{document}